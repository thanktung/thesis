\chapter*{MỞ ĐẦU}
\addcontentsline{toc}{chapter}{{\bf MỞ ĐẦU}}

\section*{Lý do chọn đề tài}
Trong lĩnh vực sản xuất thiết bị điện tử, yêu cầu về chất lượng sản phẩm ngày càng khắt khe. 
Các lỗi nhỏ như lệch linh kiện, thiếu linh kiện, sai hướng, sai giá trị hay lỗi hàn đều có thể gây hỏng bo mạch và ảnh hưởng trực tiếp đến độ tin cậy của thiết bị cũng như sản phẩm. 
Hiện nay, nhiều dây chuyền vẫn kiểm tra thủ công hoặc sử dụng các hệ thống AOI/AVI thương mại có chi phí rất cao và khó tùy biến theo từng model sản phẩm. 

Vì vậy, việc nghiên cứu một hệ thống tự động phát hiện và phân loại lỗi sản phẩm SMT có tính ứng dụng linh hoạt, chi phí hợp lý và dễ triển khai thực tế là hoàn toàn cần thiết.

\section*{Tính cấp thiết}
Kiểm tra thủ công không còn đáp ứng được yêu cầu về tốc độ và độ chính xác của dây chuyền sản xuất hiện đại. Bên cạnh đó, chi phí đầu tư cho các máy AOI thương mại cao khiến doanh nghiệp khó chủ động công nghệ. 

Xu hướng công nghiệp 4.0 thúc đẩy việc ứng dụng thị giác máy và trí tuệ nhân tạo vào kiểm tra chất lượng, vì vậy một hệ thống phát hiện lỗi SMT do chính người dùng tự phát triển sẽ giúp nâng cao năng lực công nghệ và tiết kiệm chi phí. Đây chính là lý do đề tài mang tính cấp thiết cả về mặt công nghệ và thực tiễn.

\section*{Mục tiêu và nội dung nghiên cứu}
\textbf{Mục tiêu:}
\begin{itemize}
    \item Xây dựng hệ thống tự động thu thập hình ảnh và phát hiện lỗi sản phẩm SMT.
    \item Ứng dụng thuật toán thị giác máy và AI để phát hiện, phân loại và đánh giá lỗi.
    \item Đảm bảo hệ thống có tốc độ xử lý nhanh, chính xác và dễ mở rộng.
\end{itemize}

\textbf{Nội dung nghiên cứu:}
\begin{itemize}
    \item Phân tích các dạng lỗi phổ biến trong công nghệ gắn linh kiện SMT.
    \item Thiết kế mô hình phần cứng.
    \item Xử lý ảnh.
    \item Áp dụng các thuật toán của thị giác máy tính và mô hình AI để tìm ra giải pháp phù hợp.
    \item Xây dựng phần mềm kiểm tra và giao diện hiển thị.
    \item Đánh giá hiệu năng và độ chính xác của hệ thống.
\end{itemize}

\section*{Phương pháp nghiên cứu}
\begin{itemize}
    \item Nghiên cứu lý thuyết: thị giác máy, xử lý ảnh, AI, các thuật toán đối sánh và phân loại lỗi.
    \item Thực nghiệm: thu thập dữ liệu, chụp mẫu, huấn luyện mô hình và kiểm thử với nhiều trường hợp lỗi.
    \item Phân tích – so sánh: đánh giá hiệu năng các thuật toán để chọn phương pháp tối ưu.
    \item Xây dựng và kiểm chứng mô hình: triển khai thực tế trên fixture hoặc dây chuyền SMT (nếu có).
\end{itemize}

\section*{Ý nghĩa khoa học và thực tiễn}
\textbf{Ý nghĩa khoa học:}
\begin{itemize}
    \item Đề tài áp dụng kết hợp thuật toán thị giác máy truyền thống và mô hình AI vào bài toán kiểm tra SMT.
    \item Đưa ra quy trình kiểm tra tối ưu.
\end{itemize}

\textbf{Ý nghĩa thực tiễn:}
\begin{itemize}
    \item Cung cấp giải pháp chi phí thấp, dễ tùy biến theo từng sản phẩm.
    \item Giảm lỗi con người, tăng năng suất và nâng cao chất lượng sản phẩm.
    \item Giúp doanh nghiệp chủ động công nghệ, giảm phụ thuộc vào hệ thống ngoại nhập.
\end{itemize}
