\centering
\textbf{HỆ THỐNG GIÁM SÁT VÀ PHÂN LOẠI LỖI PCB SỬ DỤNG YOLO VÀ ROBOT CÔNG NGHIỆP}

\centering \textbf{\small Vương Thanh Tùng} \\ \vspace{6pt}

\centering \textit{\small Khóa QH 2021-I/CQ, Ngành Kỹ thuật điều khiển và Tự động hóa} \\ \vspace{18pt}

\justifying
\noindent\textbf{Tóm tắt đồ án tốt nghiệp:}

Trong sản xuất điện tử, kiểm tra ngoại quan linh kiện SMD trên bảng mạch PCB sau công đoạn SMT đóng vai trò quan trọng nhằm đảm bảo chất lượng sản phẩm và giảm thiểu lỗi phát sinh. Tuy nhiên, phương pháp kiểm tra thủ công vẫn tồn tại nhiều hạn chế như thời gian kiểm tra dài, chi phí nhân công lớn và độ chính xác phụ thuộc vào người vận hành. 

Để khắc phục các vấn đề trên, đồ án đề xuất một hệ thống giám sát và phân loại lỗi PCB dựa trên mô hình học sâu YOLO kết hợp với quy trình tiền xử lý và tăng cường dữ liệu nhằm nâng cao khả năng nhận diện trong điều kiện thay đổi về ánh sáng và hình học. Hệ thống sử dụng camera thu nhận hình ảnh, mô hình YOLO để phát hiện và phân loại linh kiện lỗi, cùng với một trạm tự động hóa bao gồm cảm biến, cơ cấu chấp hành và robot công nghiệp để thực hiện thao tác phân loại sản phẩm lỗi theo thời gian thực.

Kết quả thử nghiệm cho thấy mô hình đạt độ chính xác cao trong phát hiện sai lệch linh kiện SMD và hệ thống tích hợp hoạt động ổn định, có khả năng áp dụng như một trạm kiểm tra tự động trong dây chuyền sản xuất SMT.

\noindent \textit{\textbf{Từ khóa:} YOLO, Computer Vision, PLC, Robot, Python}
