\chapter{Giới thiệu chung}

\section{Bối cảnh và xu hướng phát triển}

\section{}

Nhu cầu về sản xuất chất lượng cao, hiệu suất lớn và mang tính cá nhân hóa đang tăng mạnh trong thời đại Công nghiệp 4.0, qua đó nhấn mạnh vai trò quan trọng của hệ thống kiểm tra bằng thị giác máy.

Thị giác máy, một thành phần của trí tuệ nhân tạo (AI), cho phép kiểm soát 100\% quy trình sản xuất với tốc độ rất cao. Bằng việc sử dụng các thuật toán học máy để nhận dạng hình ảnh, hệ thống có thể giám sát toàn bộ quá trình sản xuất, nâng cao độ chính xác, hiệu suất và chất lượng kiểm soát. Điều này giúp loại bỏ lỗi chủ quan và tình trạng mệt mỏi dễ xảy ra trong kiểm tra thủ công, duy trì quá trình vận hành liên tục và ổn định.

Các ngành như dược phẩm và chế biến thực phẩm phụ thuộc vào hệ thống kiểm tra bằng hình ảnh để đảm bảo tính đồng nhất của sản phẩm và đáp ứng các tiêu chuẩn an toàn. Dưới đây là cách các hệ thống kiểm tra thị giác máy tự động tiên tiến đang định hình ngành sản xuất, cùng các lợi ích chính mà chúng mang lại cho cơ sở sản xuất.

