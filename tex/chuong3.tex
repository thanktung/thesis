\chapter{Xây dựng hệ thống và chương trình điều khiển}
% =======================
% PHẦN 1: SƠ ĐỒ KHỐI
% =======================
\section{Sơ đồ khối hệ thống}
Thành phần hệ thống bao gồm một bộ PLC bao gồm nguồn, CPU, Module QJ61BT11N, Module QD77MS16; một màn hình HMI; các remote IO và các trạm Robot; các động cơ servo cùng bộ khuếch đại; cuối cùng là một máy tính sử dụng để triển khai hệ thống thị giác máy tính.
Sơ đồ khối hệ thống như Hình \ref{c3:struc}:

\begin{figure}[h]
    \centering
    \includegraphics[width=0.75\textwidth]{figures/Chapter3/structure.png}
    \caption{Sơ đồ khối hệ thống}
    \label{c3:struc}
\end{figure}

% =======================
% PHẦN 2: THUẬT TOÁN ĐIỀU KHIỂN
% =======================
\section{Thuật toán điều khiển}

Chưa làm/////

Giải thích chi tiết cách điều khiển của hệ thống.


% =======================
% PHẦN 3: CẤU HÌNH HỆ THỐNG
% =======================
\section{Cấu hình hệ thống}
%====================================================================================

\subsection{Cấu hình hệ thống PLC và module}
Công cụ lập trình dùng cho PLC Mitsubishi Q Series là phần mềm GX Works 2.
Khi khởi tạo project cần chọn dòng PLC (Q mode); loại CPU (Q13UDV); loại project (Simple Project) và ngôn ngữ lập trình sử dụng trong project (Ladder). Cấu hình như Hình \ref{c3:newprj}.
\begin{figure} [h]
    \centering
    \includegraphics[width=0.75\linewidth]{figures/Chapter3/newprj.png}
    \caption{Cấu hình New Project}
    \label{c3:newprj}
\end{figure}

Sau khi đã tạo thành công project mới, cần cấu hình hệ thống PLC tương ứng với phần cứng sử dụng. 
\begin{enumerate}
    \item Cấu hình base unit và các module giống với thiết lập phần cứng như Hình \ref{c3:hardware}. 
    Thiết lập thông tin của base unit và vị trí module trong GX Works2 như Hình \ref{c3:io} tại \textit{Parameter} $\rightarrow$ \textit{PLC Parameter} $\rightarrow$ \textit{I/O Assignment}.
    \begin{figure}[H]
    \centering
    \includegraphics[width=0.75\linewidth]{figures/Chapter3/phancung.png}
    \caption{Thiết lập phần cứng}
    \label{c3:hardware}
    \end{figure}

    
    \begin{figure}[H]
        \centering
        \includegraphics[width=0.75\linewidth]{figures/Chapter3/io.png}
        \caption{IO Assignment}
        \label{c3:io}
    \end{figure}

    \item Cấu hình giao tiếp/kết nối giữa PLC với GX Works2 (PC Side) như Hình \ref{c3:ethernet}. Bao gồm thiết lập IP cho PLC, giao thức MC Protocol.
    \begin{figure}[H]
        \centering
        \includegraphics[width=0.75\linewidth]{figures/Chapter3/ethernet.png}
        \caption{Thiết lập kết nối Ethernet}
        \label{c3:ethernet}
    \end{figure}  
\end{enumerate}
Sau khi kết thúc thiết lập, bấm End để lưu thiết lập.

%-----=======================================================================================
\subsection{Cấu hình CC-Link giao tiếp giữa PLC và Station}
Để có thể kết nối các Stations với PLC qua CC-Link cần thiết lập cả phần cứng và phần mềm.
Cần cấu hình hai thông số quan trọng là phiên bản CC-Link (Ver 2 hoặc Ver 1), tốc độ truyền (khoảng cách càng gần thì tốc độ càng cao), số thứ tự Station cho từng trạm (cả Remote IO và Remote Device).
\begin{enumerate}
    \item Trạm I/O: Cấu hình các trạm IO là Remote IO Station. 
    Hình minh họa phần cứng của Remote IO giống như Hình\ref{c3:remoteio}. Có hai thành phần quan trọng là công tắc định địa chỉ Station (3) và công tắc định tốc độ giao tiếp (2). Cần phải điều khiển sao cho tốc độ đúng với tốc độ hệ thống và địa chỉ phù hợp, không bị trùng với các trạm khác.
        
    \begin{figure}[H]
        \centering
        \includegraphics[width=0.75\linewidth]{figures/Chapter3/remote-io.png}
        \caption{Phần cứng Remote IO}
        \label{c3:remoteio}
    \end{figure}  

    \item Robot Hyundai: Cấu hình robot là Remote Device Station.
    Để có thể giao tiếp với mạng CC-Link, cần gắn thêm board giao tiếp CC-Link BD570 vào bộ điều khiển, minh họa như Hình\ref{c3:cc1}.
    \begin{figure}[H]
        \centering
        \includegraphics[width=0.75\linewidth]{figures/Chapter3/h-cc.png}
        \caption{Board CC-Link}
        \label{c3:cc1}
    \end{figure}
    Gắn board CC-Link vào bộ điều khiển, sau đó thiết lập vị trí trạm và tốc độ truyền bằng các switch có trên board.
    Sau khi bật nguồn bộ điều khiển, thông tin về giao tiếp CC-Link được hiển thị bằng Teach pendant (Hình\ref{c3:cc2}).
     \begin{figure}[H]
        \centering
        \includegraphics[width=0.75\linewidth]{figures/Chapter3/h-cc2.jpg}
        \caption{Thông tin thiết lập CC-Link}
        \label{c3:cc2}
    \end{figure}

    \item Robot Yaskawa: Cấu hình robot là Remote Device Station 
    Để có thể kết nối Robot với PLC qua CC-Link giúp truyền nhận tín hiệu In/Out và Register, cần phải lắp đặt phần cứng là CCS-PCIE Board vào bộ điều khiển của Robot (YRC1000micro), kết nối dây theo tiêu chuẩn của CC-Link và cấu hình Robot như một Remote Device Station. Minh họa như Hình \ref{c3:yas1}
    \begin{figure}[H]
        \centering
        \includegraphics[width=0.5\linewidth]{figures/Chapter3/yas1.png}
        \caption{Kết nối CC-Link giữa Robot Yaskawa và PLC}
        \label{c3:yas1}
    \end{figure}

    Board CCS-PCIE (Hình \ref{c3:yas2})
    \begin{figure}[H]
        \centering
        \includegraphics[width=0.5\linewidth]{figures/Chapter3/y-cc1.png}
        \caption{Board CCS-PCIE}
        \label{c3:yas2}
    \end{figure}

    Thiết lập giao tiếp CC-Link cần vào Maintenance Mode: Nhấn [MAIN MENU] trong khi bật nguồn YRC1000micro. Sau khi vào được Maintenance mode cần vào thiết lập thông tin cho Board CCS-PCIE:
    [MAIN MENU] -> SYSTEM -> SETUP -> OPTION BOARD -> CCS-PCIE.
    Thiết lập thông số cho board CCS-PCIE như dưới:
    \begin{figure}[H]
        \centering
        \includegraphics[width=0.5\linewidth]{figures/Chapter3/y-cc2.png}
        \caption{Thiết lập CC-Link}
        \label{c3:yas3}
    \end{figure}
    Ấn [ENTER], cho đến màn hình EXTERNAL IO SETUP. Đây là phần thiết lập dữ liệu gửi và nhận (Rx, Ry) trên mạng CC-Link được đưa vào External Input và External Output.
    \begin{figure}[H]
        \centering
        \includegraphics[width=0.5\linewidth]{figures/Chapter3/y-cc3.png}
        \caption{Controller Information}
        \label{c3:yas4}
    \end{figure}

    Hệ thống có ba loại dữ liệu vào/ra như sau:

\begin{enumerate}
    \item \textbf{Dữ liệu từ board ASF30} (board I/O của bộ điều khiển, gồm 8-bit input và 8-bit output):
    \begin{itemize}
        \item 8-bit input: \#20010~--~\#20017
        \item 8-bit output: \#30010~--~\#30017
    \end{itemize}

    \item \textbf{Dữ liệu trạng thái từ board CCS-PCIE} (chiếm 8-bit input và 8-bit output):
    \begin{itemize}
        \item 8-bit input: \#20020~--~\#20027
        \item 8-bit output: \#30020~--~\#30027
    \end{itemize}

    \item \textbf{Dữ liệu truyền qua mạng CC-Link} (112-bit input và 112-bit output):
    \begin{itemize}
        \item 14-byte input: \#20030~--~\#20167
        \item 14-byte output: \#30030~--~\#30167
    \end{itemize}
\end{enumerate}

    Sau khi thiết lập dữ liệu được gửi từ CC-Link sang Robot, có thể kiểm tra trạng thái các bit truyền/nhận tại:
    \texttt{[MAIN MENU] → IN/OUT → EXTERNAL INPUT/OUTPUT}

    Giá trị của các bit \texttt{Rx} và \texttt{Ry} có thể xem trong \textit{External Input/Output}.  
    Tuy nhiên, các bit được sử dụng trong \textit{Job} là các bit thuộc \textit{General Purpose Input/Output}.  
    Do đó, cần chuyển các bit từ \textit{External Input/Output} sang \textit{General Purpose Input/Output} thông qua \textbf{Ladder Program}.

    Để chỉnh sửa \textit{User Ladder Program}, vào:
    \texttt{[MAIN MENU] → IN/OUT → LADDER PROGRAM}
    Chọn:
    \texttt{Display → User Program}
    để thực hiện chỉnh sửa Ladder (Hình\ref{c3:yas5}).
\begin{figure}[H]
    \centering
    \includegraphics[width=0.5\linewidth]{figures/Chapter3/yas4.png}
    \caption{Ladder Program}
    \label{c3:yas5}
\end{figure}
\begin{figure}[H]
    \centering
    \includegraphics[width=0.5\linewidth]{figures/Chapter3/yas5.png}
    \caption{Convert Data}
    \label{c3:yas6}
\end{figure}

    Ví dụ: tại dòng 98--99, tương ứng với \texttt{Step 0026}, ý nghĩa là chuyển \textbf{1-byte} dữ liệu từ 
    \texttt{\#20030~--~\#20037} của \textit{External Input} vào 
    \texttt{\#00030~--~\#00037} của \textit{General Purpose Input}. 

    Điều này tương đương với: Điều này tương đương với: \(\texttt{IN\#0017} \rightarrow \texttt{IN\#0024}\).

    \item Cấu hình PLC là Master:
    Sau khi đã cấu hình xong các trạm Remote I/O và Remote Device Station, tiếp đến là cấu hình giao tiếp CC-Link tại trạm Master (module QJ16BT11N). 
    Dữ liệu truyền qua CC-Link có 4 dạng là: Rx (tín hiệu đầu vào), Ry (tín hiệu đầu ra), RWw (ghi word 16-bit ), RWr (đọc word 16-bit).
    Trong Tab Parameter / Network parameter / CC-Link của Project trong GX-Works2, cấu hình như Hình\ref{c3:plc1}:
\begin{figure}[H]
    \centering
    \includegraphics[width=0.5\linewidth]{figures/Chapter3/plc-cc1.png}
    \caption{Master Station}
    \label{c3:plc1}
\end{figure}
\begin{figure}[H]
    \centering
    \includegraphics[width=0.5\linewidth]{figures/Chapter3/plc-cc2.png}
    \caption{Master Station Setup}
    \label{c3:plc2}
\end{figure}
\begin{figure}[H]
    \centering
    \includegraphics[width=0.5\linewidth]{figures/Chapter3/plc-cc3.png}
    \caption{CC-Link Speed}
    \label{c3:plc3}
\end{figure}

\end{enumerate}

Sau khi thiết lập xong Master, thu được kết quả như sau:

\begin{itemize}
    \item \textbf{Station \#1:} Module Input \texttt{AJ65SBTB1-32D}
    \item \textbf{Station \#2:} Module Output \texttt{AJ65SBTB1-32T1}
    \item \textbf{Station \#3:} Robot Yaskawa (chiếm 4 trạm)
    \item \textbf{Station \#4:} Robot Hyundai (chiếm 4 trạm)
    \item \textbf{Station \#5:} \texttt{AJ65SBT1-16DT1}
\end{itemize}

Mỗi trạm 32~bit chiếm:
\begin{itemize}
    \item \textbf{32 bit} tại \texttt{Rx}, \texttt{Ry}
    \item \textbf{4 Word} tại \texttt{RWr}, \texttt{RWw}
\end{itemize}

Bộ nhớ tại trạm Master được phân bổ cho từng Slave như sau:

\begin{center}
\begin{tabular}{|c|c|c|c|c|}
\hline
\textbf{Station} & \textbf{Rx} & \textbf{Ry} & \textbf{RWr} & \textbf{RWw} \\ \hline

\#1 & X500--X51F & Y4E0--Y4FF & D40--D43   & D100--D103 \\ \hline
\#2 & X520--X53F & Y500--Y51F & D44--D47   & D104--D107 \\ \hline
\#3 & X540--X5BF & Y520--Y59F & D48--D63   & D108--D123 \\ \hline
\#4 & X5C0--X63F & Y5A0--Y61F & D64--D79   & D124--D139 \\ \hline
\#5 & X640--X65F & Y620--Y63F & D80--D83   & D140--D143 \\ \hline

\end{tabular}
\end{center}



%=------------------------------------------------------------------------------
\subsection{Cấu hình KepserverEX làm server OPC UA}
Kết nối KepserverEX với PLC qua card mạng của máy tính. Giao thức được chọn và sử dụng trong phần mềm GX-Works2 và phần mềm KepserverEX phải giống nhau để đảm bảo giao tiếp giữa OPC UA Server với PLC.

\begin{figure}[H]
        \centering
        \includegraphics[width=0.75\linewidth]{figures/Chapter3/kepwareprj.png}
        \caption{Cấu hình Project Kepware}
        \label{c3:kepwareprj}
 \end{figure}

\begin{figure}[H]
        \centering
        \includegraphics[width=0.75\linewidth]{figures/Chapter3/kepwarechannel.png}
        \caption{Cấu hình Channel}
        \label{c3:kepwarechannel}
 \end{figure}

\begin{figure}[H]
        \centering
        \includegraphics[width=0.75\linewidth]{figures/Chapter3/kepwaredevice.png}
        \caption{Cấu hình Device}
        \label{c3:kepwaredevice}
 \end{figure}


\subsection{Cấu hình GigE Camera Basler ac2500}
Kết nối Camera với máy tính qua cổng mạng IP, cần cấu hình IP của camera với IP của máy tính trong cùng lớp mạng.

\begin{figure}[H]
        \centering
        \includegraphics[width=0.75\linewidth]{figures/Chapter3/baslerip.png}
        \caption{Cấu hình IP camera}
        \label{c3:cameraip}
 \end{figure}

Tiếp theo là cấu hình thông số ảnh chụp như Gain, Gamma, Exposure time. Có thể sử dụng chức năng Automatic Image Adjustment.

\begin{figure}[H]
        \centering
        \includegraphics[width=0.75\linewidth]{figures/Chapter3/baslerauto.png}
        \caption{Cấu hình ảnh chụp}
        \label{c3:image}
 \end{figure}


% =======================
% PHẦN 3: XỬ LÝ ẢNH
% ======================= 

\section{Xây dựng hệ thống xử lý ảnh}

\subsection{Pipeline xử lý ảnh}
Hệ thống kiểm tra ngoại quan tự động (Automated Visual Inspection – AVI) vận hành thông qua một chuỗi các bước xử lý ảnh được tổ chức theo dạng pipeline nhằm đảm bảo quá trình kiểm tra diễn ra chính xác, ổn định và nhất quán. Pipeline xử lý ảnh của hệ thống có thể được mô tả qua các giai đoạn chính như sau:

\subsubsection{Thu nhận hình ảnh}
Ở giai đoạn đầu tiên, camera độ phân giải cao hoặc thiết bị ghi hình chuyên dụng được sử dụng để thu nhận ảnh hoặc video của sản phẩm. Chất lượng ảnh đầu vào phụ thuộc vào loại camera, ống kính, khoảng cách chụp và cấu hình hệ thống chiếu sáng. Đây là nguồn dữ liệu cơ bản phục vụ toàn bộ quá trình xử lý phía sau.

\subsubsection{Xử lý và tiền xử lý hình ảnh}
Sau khi thu nhận, hình ảnh được đưa vào các thuật toán xử lý nhằm tăng cường chất lượng, làm nổi bật đặc trưng quan trọng và giảm nhiễu. Các bước thường bao gồm:
\begin{itemize}
    \item Cân bằng sáng, tăng tương phản hoặc hiệu chỉnh màu.
    \item Lọc nhiễu và cải thiện độ nét.
    \item Chuẩn hóa kích thước, góc xoay hoặc cắt vùng quan tâm (ROI).
\end{itemize}
Giai đoạn này đảm bảo rằng hình ảnh đầu vào luôn đạt chất lượng ổn định để phục vụ nhận diện lỗi.

\subsubsection{Phân tích ảnh và nhận diện lỗi}
Ở bước này, hệ thống áp dụng các thuật toán xử lý ảnh nâng cao và mô hình học máy để:
\begin{itemize}
    \item Trích xuất đặc trưng của sản phẩm.
    \item Phát hiện các bất thường hoặc khuyết tật như vết nứt, thiếu linh kiện, sai lệch vị trí, biến dạng hình học.
    \item So sánh với hình mẫu hoặc tiêu chuẩn kỹ thuật đã thiết lập.
\end{itemize}
Thuật toán AI và machine learning giúp hệ thống thích nghi với biến động trong môi trường sản xuất, từ đó cải thiện độ chính xác theo thời gian.

\subsubsection{Ra quyết định}
Dựa trên các tiêu chí chất lượng đã định nghĩa trước, hệ thống phân loại sản phẩm thành hai nhóm: đạt yêu cầu (OK) hoặc không đạt (NG). Việc ra quyết định được tự động hóa và đảm bảo tính khách quan, loại bỏ hoàn toàn yếu tố chủ quan của con người.

\subsubsection{Phản hồi và lưu trữ dữ liệu}
Kết quả kiểm tra được ghi nhận và phản hồi trực tiếp đến hệ thống sản xuất để thực hiện các hành động kịp thời như loại bỏ sản phẩm lỗi hoặc điều chỉnh quy trình. Đồng thời, dữ liệu hình ảnh, log lỗi và thống kê được lưu trữ trên cơ sở dữ liệu cục bộ hoặc đám mây để phục vụ phân tích chất lượng, tối ưu hoá dây chuyền và truy vết trong sản xuất.

Pipeline trên giúp hệ thống AVI hoạt động mạch lạc, hiệu quả và có khả năng mở rộng theo yêu cầu thực tế của từng dây chuyền sản xuất.

\subsection{Train mô hình YOLO detect linh kiện SMD}

\subsubsection{Thu thập dữ liệu}
Ảnh được thu thập trực tiếp từ camera trong dây chuyền bằng cách chụp nhiều mẫu PCB ở nhiều điều kiện khác nhau: thay đổi độ sáng, vị trí, góc đặt bảng, và trạng thái linh kiện (đầy đủ hoặc lỗi). Mục tiêu là tạo bộ dữ liệu đa dạng, phản ánh đúng môi trường sản xuất thực tế để mô hình học sâu có khả năng tổng quát tốt.

\begin{figure}[H]
    \centering
    \includegraphics[width=0.75\linewidth]{figures/Chapter3/image11.png}
    \caption{Thu thập ảnh}
    \label{c3:acquistion}
\end{figure}

\subsubsection{Gán nhãn dữ liệu (Labeling)}
Sau khi thu thập, toàn bộ ảnh được đưa vào nền tảng Roboflow để thực hiện gán nhãn. Mỗi linh kiện SMD hoặc vùng lỗi trên PCB được bao quanh bằng bounding box và gán class tương ứng, chẳng hạn như \texttt{OK}, \texttt{Missing}, \texttt{Shifted}, \texttt{Wrong\_Orientation}, \ldots{} Việc gán nhãn được thực hiện thủ công nhằm đảm bảo độ chính xác, vì chất lượng nhãn có ảnh hưởng trực tiếp đến khả năng học của mô hình YOLO.

Roboflow hỗ trợ việc chuẩn hoá kích thước ảnh, tăng cường dữ liệu (augmentation) như xoay, thay đổi độ sáng, cắt, lật ảnh, giúp tăng độ đa dạng cho tập huấn luyện. Sau khi hoàn tất, bộ dữ liệu được export theo định dạng YOLO và sẵn sàng cho quá trình huấn luyện.
\begin{figure}[H]
    \centering
    \includegraphics[width=0.75\linewidth]{figures/Chapter3/image12.png}
    \caption{Label ảnh bằng Roboflow}
    \label{c3:labeling}
\end{figure}

\subsubsection{Huấn luyện mô hình (Train model)}
Bộ dữ liệu sau khi export được đưa lên môi trường Google Colab để huấn luyện mô hình YOLO. Colab cung cấp GPU miễn phí, phù hợp cho huấn luyện nhanh các mô hình kích thước vừa.

Quy trình huấn luyện gồm các bước:
\begin{itemize}
    \item Upload bộ dữ liệu từ Roboflow vào Colab bằng API.
    \item Cấu hình file huấn luyện: đường dẫn dữ liệu, số lớp (classes), độ phân giải ảnh, batch size, và số epochs.
    \item Tiến hành huấn luyện và theo dõi các chỉ số: loss, precision, recall, mAP.
    \item Lưu lại \texttt{best.pt} sau khi huấn luyện, sử dụng cho giai đoạn suy luận (inference) trong hệ thống AVI.
\end{itemize}

Mô hình sau khi huấn luyện được tải xuống và đưa vào pipeline xử lý ảnh để thực hiện phát hiện linh kiện SMD trong thời gian gần thực.




